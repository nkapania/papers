% Document settings
\documentclass[11pt]{article}
\usepackage[margin=1in]{geometry}
\usepackage{graphicx}
\usepackage{multirow}
\usepackage{setspace}
\usepackage{amsmath,amssymb,gensymb,epsfig, epstopdf}
\usepackage{xcolor}
\pagestyle{plain}
\setlength\parindent{0pt}
\setlength{\parskip}{1em}

\begin{document}

% Header
\textbf{Nitin Kapania} \\
\textbf{SciRob Intro - Outline} \\

\begin{enumerate}
	\item Autonomous vehicles rely on control systems to automate following of a desired trajectory. These control systems are presently designed around a physical model of the vehicle. 

	\subitem Include references to several overview papers on path planning / control
	for autonomous vehicles
	
	\subitem Control systems for autonomous vehicles currently rely on a  physics-based model either explicitly in the case of predictive control algorithms or implicitly for traditional feedforward-feedback architectures
	
	\subitem Control systems based on physical models have clear advantages and have demonstrated great performance under known driving conditions. 

	\subitem Mention advantages of physics-based models: simplicity in terms of required parameters, intuition, and correctness from known first principles.  

	\item However, physics-based models have several drawbacks, particularly when operating conditions change or when dynamics are difficult to model. 

	\subitem Physics-based models are designed around a particular set of parameters and a particular operating condition. These can often change in safety critical situations. Provide example of car designed around asphalt that suddenly encounters a low-friction surface. 

	\subitem Designer must explicitly decide about model complexity and manually encode logic for multiple operating conditions.  

	\subitem Certain physics that are critical for control can be prohibitive to model given difficulty of estimating parameters or hard-to-model physics (e.g. effect of suspension dynamics at the handling limits)  

	\item Given drawbacks of physics-based models, there has been increasing interest in developing data-driven models for automated systems. 

	\subitem Reference prior art here - e.g. Punjani and Abbeel, Cole, Ghazizadeh, and papers that don't use neural networks (e.g. regression models, GPs, etc.)

	\subitem Research with neural networks dates back to early 1990's, but  explosion of deep learning methods has created potential for models with even higher capacity

	\item This paper contributes a novel neural network model for an automated vehicle. 

	\subitem Describe contributions and differentiation from prior art 

\end{enumerate}


\end{document}

