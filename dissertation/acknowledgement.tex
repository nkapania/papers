% the last preface section (e.g., acknowledgement.tex)
% should look like
\prefacesection{Acknowledgment}

My dad has been my biggest role model for as long as I can remember,
dating back to when I was three or four and liked to copy everything he did. There's a picture we have back home where my dad is
shaving before work, and there I am standing next to him with shaving cream all over my face, trying to imitate him with a (hopefully) fake razor 
my mom gave me. I think it was in middle school when I first comprehended how smart my dad was, just based on the kinds of books
he had in his office at work and home. I would sometimes open one up just to see the fancy math symbols, which
were like a beautiful foreign language to me. Wanting to be like him is probably the number one reason I went to an engineering school
for undergrad and definitely a major reason why I decided to do a PhD. But he's a statics guy, dealing primarily with structures that really should
not move too much. Sorry dad, but I've always found dynamics a little more exciting, and cars are just more beautiful than planes ;)

The rest of my family is pretty awesome too. My younger sister Esha has an enormous heart and in her words, is the ``glue" that holds
our family together. While I would like her to spend a little less time studying in med school and more time out and about in Chicago,
I always look forward to catching up with her. My youngest sister Rhea has been gifted with a sense of humor very similar to mine, and even though
she is eight years younger than me, we have a blast playing video games, quoting \textit{Mean Girls}, and constantly teasing Esha. Finally, my mom has
been the best mom a son could have, and has supported me even though she was very sad when her ``bestest" decided to move across the country to California. Whenever I visit home, I always
get to pick what the family eats for dinner and I always get fed first, much to my sisters' dismay.

Academically and professionally, I owe a lot to my advisor Chris Gerdes. Dedicating an entire lab to automotive research is
 very difficult given the cyclical nature
of the automotive industry. Nevertheless, he has established one of the strongest automotive research labs in the
country, and it has been amazing to be a part of his program's growth over the last five years. My favorite interactions with Chris came during
our research trips to the Thunderhill race course. At the race track, he was always eager sit in the car with me while I tested my research ideas, offering
insights and perspectives that he has developed over 25 years working with cars. In the evening, it was great to relax after a long day of data
collection and brainstorm ideas to try for the next day. I occasionally had the chance to drive home with Chris, and I always enjoyed 
our conversations about everything from the stock market to Stanford football. It was during these conversations that I realized Chris' passion
for automobiles is exceeded only by his dedication to his family, and that has had a big impact on me. 

I also owe Chris for choosing a great bunch of people for me to work with. The Dynamic Design Lab is a diverse collection of highly intelligent
people from all over the place, both geographically and personally. Some of my closest connections at Stanford have been with members of the DDL, and
I will always remember our happy hours, celebrations, and send-off parties as some of the best times I have had here. 
One thing I've learned while here is that members of our lab form a small but out-sized network of close friends and future colleagues that remains in place long after graduation. 
 In addition to being great friends and colleagues, members of the DDL are great sources of knowledge, and the ideas that are generated in the lab every day provide a great headwind for
doing amazing research. 

There are a few people I have worked with that I would like to acknowledge personally. John Subosits, Vincent Laurense, Paul Theodosis, Joe Funke and Krisada Kritiyakirana have
all been great colleagues on the Audi-Stanford racing project, and have spent many hours of their own time helping me with the
significant data collection effort required for this dissertation. Additionally, I am grateful to have had the help of Samuel Schacher and John Pedersen during the summers
of 2014 and 2015. I would also like to thank members of the Audi Electronics Research Laboratory, especially Rob Simpson and Vadim Butakov, for being great resources with
 our experimental Audi TTS testbed. In my time here, we have managed to break what seems like every component of the car at least once and went through a major overhaul of the control
 electronics and firmware. Rob and Vadim were there every time we needed them, and we never had to miss a testing a trip due to a repair that was not performed on time. Finally,
 it takes a lot of staff working behind the scenes to do great research, and I would like to thank Erina and Jo for the great job they have done over the last two years.
 
I also would like to thank Dr. Mykel Kochenderfer and Dr. Allison Okamura for helping me strengthen this dissertation. Allison started at Stanford in 2011 just as I did, and I've always felt that I 
could go to her for anything I needed help with. I also became friends with many of her students in the CHARM lab, and it was great to relax with them
during design group happy hours. Dr. Kochenderfer arrived at Stanford during my fourth year, and has not only helped me structure my thesis contributions clearly
and concisely, but has become a welcome addition to the autonomous vehicle program at Stanford. 

And finally, where would I be without my lovely girlfriend Margaret? I met Margaret during my first year as a graduate student, and she has been a 
great companion over the last four years as we have explored graduate life and the Bay Area together. Margaret likes to say that she followed me into
doing a PhD, but the truth is that I have been following her for a lot longer. Margaret is a true believer in enjoying the everyday beauty of life, 
and has showed me me how to enjoy things as simple as going on a hike, relaxing with her newly adopted cat, or more recently, eating at the same burger joint
every single Saturday night ;) I'm not sure what my life will be like after I leave Stanford, but I know Margaret will be a significant part of it, and that
makes me feel A-OK.